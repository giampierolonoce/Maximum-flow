\documentclass{article}
\title{Tesi}
\author{}
\date{}
\usepackage{amsthm}
\usepackage[utf8x]{inputenc}
\usepackage{amsmath}
\usepackage{amssymb}
\usepackage{amsfonts}
\usepackage{mathrsfs}
\usepackage{graphicx}
\usepackage{floatflt}
\usepackage{amscd}
\usepackage{epigraph}
\usepackage{verbatim}
\theoremstyle{plain}                    
\newtheorem{teo}{Theorem}[section]      
\newtheorem{prop}[teo]{Proposizione}    
\newtheorem{cor}[teo]{Corollary}       
\newtheorem{lem}[teo]{Lemma}            
\theoremstyle{definition}               
\theoremstyle{remark}                  
\newtheorem{oss}{Observation}          
\usepackage[colorlinks=true,linkcolor=blue]{hyperref}

\begin{document}
Preliminar notes on maximum-flow problem solved in FCPP.\\
Notation:\\
 $\small\bullet\ G$ will denote the directed weighted graph of capacities, 
 $s$ and $t$ will denote respectively source and sink of our graph.\\
We will assume that capacities are nonnegative and symmetrical.\\
 $\small\bullet\ $  For a path $\mathfrak{p}$ we will denote with $|\mathfrak{p}|$ its lenght.\\

When it will be convenient, we will treat a directed weighted graph $G$ as a function $G: V\times V\ \rightarrow \mathbb{R_{+}}$, where $V$ is its set of vertices. With a little abuse of notation we will say that the edge from a node $\delta$ to a node $\delta'$ is in $G$, or $(\delta, \delta') \in G$,  in place of $G(\delta, \delta') \not= 0$.\\
Likewise we will say that a path $\mathfrak{p}$ is contained in $G$, or $\mathfrak{p}\subset G$, if every edg                                                                                                                                                                                                                                                                                                                                                                                                                                                                                                                                                                                                                                                                                                                                                                                                                                                                                                                                                                                                                                                                                                                                                                                                                                                                                                                                                                                                                                                                                                                                                                                                                                                                                                                                                                                                                                                                                                                                                                                                                                                                                                                                                                                                                                                                                                                                                                                                                                                                                                                                                                                                                                                                                                                                                                                                                                                                                                                                                                                                                                                                                                                                                                                                                                                                                                                                                                                                                                                                                                                                                                                                                                                                                                                                                                                                                                                                                                                                                                                                                                                                                                                                                                                                                                                                                                                                                                                                              

 $\small\bullet\ $ For a field $\Phi\ , \mathcal{R}_{\Phi}$ will denote the directed weighted graph of residual capacities respect to $\Phi$:
\begin{equation}
\label{eq:residual-capacity}
\mathcal{R} =
    G-\Phi
\end{equation}

$\small\bullet\ $  We define admissible path respect to $\Phi$  a path $\gamma$ in $G$ from source to sink such that 
\begin{equation*}
\Phi|_{\gamma} < G|_{\gamma}
\end{equation*}
$\small\bullet\ $ We say that a flow $\Phi$ is maximal if there are no admissible paths in $G$ respect to $\Phi$.

\begin{prop}
$\Phi$ is a maximum flow iff $\mathcal{R}_{\Phi}$ does not have paths from source to sink.\\ \\ \\ \\
\end{prop}


For a field-value $f$ in $\delta$ we define $\displaystyle{|f|:=\sum_{\delta\sim\delta'}f(\delta')}$.
We now formalize functions involved in the algorithm. Let $\Phi$ be a field and $\delta\sim\delta'$ be  devices, we define $\Phi^*(\delta, \delta'):=\Phi(\delta', \delta)$.
By co-induction:
\begin{equation*}
\begin{split}
\Phi_0 &:= \Phi\\
\mathcal{R}_n&:= \mathcal{R}_{\Phi_n}\\
e_n(\delta)&:=
\begin{cases}
\infty &\text{if\ }\delta=s\\
-\infty &\text{if\ }\delta=t\\
|\Phi_n(\delta)| &\text{otherwise}
\end{cases}\\
d_0(\delta)&:=
\begin{cases} 
0 &\text{if\ }\delta=t\\
\infty&\text{otherwise}
\end{cases}\\
d_n(\delta)&:=
\begin{cases}
0 &\text{if\ }\delta=t\\
 \mathrm{min}\big\{d_{n-1}(\delta') + 1\ |\  \mathcal{R}_{n-1}(\delta, \delta')>0\big\}&\text{otherwise}
\end{cases}\\
I_{n+1}&:= \mathrm{trunc}(\mathcal{R}_n\cdot(d_n^*<d_{n+1}),  e(\Phi_n^*))\\
\Phi_{n+1}&:= -\Phi_n^* + I_{n+1} + \mathrm{trunc}(\Phi_n^*\ ,\  e(\Phi_n^* - I_{n+1}))\\
X_n&:=\{\delta \ |\ d_n(\delta)=\infty\}
\end{split}
\end{equation*}

\begin{lem}
 Eventually $\{X_n\}_n$ stabilizes to a set of devices $X$.
\end{lem}
\begin{proof}
 %For every device $\delta$ let $\epsilon_\delta$ the first event on $\delta$ such that $d_{\mathrm{round}(\epsilon_\delta)}(\delta) < \infty$, if there's any. Moreover let $\epsilon_0$ be an event in the future of all the $\epsilon_\delta$ and $\bar{n} = \mathrm{round}(\epsilon_0)$. \\
For every $n\in\mathbb{N}$ let's consider the set $T_n=\{\delta\ |\ \mathrm{dist}_G(\delta, t)\leq n\}$.\\
We prove by induction that for $\delta\not\in T_n $ we have $d_n(\delta)=\infty$.\\
Base step: $T_0=\{t\}$, and condition on $d_0$ holds by definition.\\
Inductive step: suppose that $\forall \delta'\not\in T_{n-1}\ d_{n-1}(\delta')=\infty$ and consider a  $\delta \not\in T_n$. We have  $d_n(\delta)= \mathrm{min}\big\{d_{n-1}(\delta') + 1\ |\  \mathcal{R}_{n-1}(\delta, \delta')>0\big\}$ and since no $\delta'\sim \delta$ is in $T_{n-1}$, condition on $d_n$ holds by inductive hypothesis.\\ \\
As a consequence, for a $\delta\in  T_n$ 
\begin{equation*}
d_n(\delta) =  \underset{\delta'\in T_n}{\mathrm{min}}\big\{d_{n-1}(\delta') + 1\ |\  \mathcal{R}_{n-1}(\delta, \delta')>0\big\}
\end{equation*}
 Now we want to show by induction that $\displaystyle{\forall n\  X_n\cap T_n \subseteq X_{n+1} \cap T_{n+1}}$.\\ 
Base step: $X_0\cap T_0 = \emptyset$ and we're done.\\
 Inductive step: let  $\delta$ be a device in $X_n\cap T_n$ and $\delta'\in T_n$ such that $\delta'\sim\delta$. \\
We have $\displaystyle{d_{n-1}(\delta')=\infty\ \lor\ \mathcal{R}_{n-1}(\delta, \delta') = 0}$. If $d_{n-1}(\delta')=\infty$ then $d_n(\delta')=\infty$ by inductive hypothesis. Otherwise let's suppose $d_{n-1}(\delta')<\infty$ and $d_n(\delta')<\infty$ 
\\ \\ \\.
\begin{equation*}
\forall\delta'\sim\delta\ \ \ \mathcal{R}_{n-1}(\delta, \delta')\cdot(d_{n-1}(\delta')<d_{n}(\delta)) = 0
\end{equation*}
and thus $I_n(\delta)$ is a zero field-value.
So we have 

\end{proof}



\end{document}