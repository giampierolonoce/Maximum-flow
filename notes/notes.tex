\documentclass{article}
\title{Jump in fabula}
\author{}
\date{}
\usepackage{amsthm}
\usepackage[utf8x]{inputenc}
\usepackage{amsmath}
\usepackage{amssymb}
\usepackage{amsfonts}
\usepackage{mathrsfs}
\usepackage{graphicx}
\usepackage{floatflt}
\usepackage{amscd}
\usepackage{epigraph}
\theoremstyle{plain}                    
\newtheorem{teo}{Theorem}[section]      
\newtheorem{prop}[teo]{Proposizione}    
\newtheorem{cor}[teo]{Corollary}       
\newtheorem{lem}[teo]{Lemma}            
\theoremstyle{definition}               
\theoremstyle{remark}                  
\newtheorem{oss}{Observation}          
\usepackage[colorlinks=true,linkcolor=blue]{hyperref}

\begin{document}
Preliminar notes on maximum-flow problem solved in FCPP.\\
Notation:\\
 $\small\bullet\ \mathcal{C}$ will denote the directed weighted graph of capacities.\\
 $s$ and $t$ will denote respectively source and sink of our graph.\\
 $\small\bullet\ \mathcal{R}$ will denote the directed weighted graph of residual capacities.\\
 $\small\bullet\ $ For a directed graph $G$, $d_G(\delta, \delta')$ will be the distance from node $\delta$ to node $\delta'$. It takes values in $\mathbb{N}\cup\{\infty \}$.\\
 $\small\bullet\ $ For a acyclic directed graph $G$, $D_G(\delta, \delta')$ will be the maximum of the lenghts of all paths from node $\delta$ to node $\delta'$.\\
 $\small\bullet\ $  For a path $\mathfrak{p}$ we will denote with $|\mathfrak{p}|$ its lenght.\\

When it will be convenient, we will treat a directed weighted graph $G$ as a function $G: V\times V\ \rightarrow \mathbb{R_{+}}$, where $V$ is its set of vertices. With a little abuse of notation we will say that the edge from a node $\delta$ to a node $\delta'$ is in $G$, or $(\delta, \delta') \in G$,  in place of $G(\delta, \delta') \not= 0$.\\
Likewise we will say that a path $\mathfrak{p}$ is contained in $G$, or $\mathfrak{p}\subset G$, if every edge in $\mathfrak{p}$ is in $G$.\\
\begin{oss} $\mathcal{C}$ is acyclic and
\begin{equation}
\forall \delta, \delta':\ \mathcal{C}(\delta, \delta')\not= 0 \Rightarrow \mathcal{C}(\delta', \delta)\ =\ 0
\end{equation}
  i.e. the capacity between any two nodes can be nonzero in almost one direction.\\
$\mathcal{R}$, has the same set of vertices than $\mathcal{C}$, it can have cycles, and has the property that 
\begin{equation}
\forall \delta, \delta':\ \mathcal{R}(\delta, \delta')+\mathcal{R}(\delta', \delta)\ =\ \mathcal{C}(\delta, \delta')+\mathcal{C}(\delta', \delta)\ .
\end{equation}
\end{oss}



\end{document}