\documentclass{article}
\title{Tesi}
\author{}
\date{}
\usepackage{amsthm}
\usepackage[utf8x]{inputenc}
\usepackage{amsmath}
\usepackage{amssymb}
\usepackage{amsfonts}
\usepackage{mathrsfs}
\usepackage{graphicx}
\usepackage{floatflt}
\usepackage{amscd}
\usepackage{epigraph}
\usepackage{verbatim}
\theoremstyle{plain}                    
\newtheorem{teo}{Theorem}[section]      
\newtheorem{prop}[teo]{Proposizione}    
\newtheorem{cor}[teo]{Corollary}       
\newtheorem{lem}[teo]{Lemma}            
\theoremstyle{definition}               
\theoremstyle{remark}                  
\newtheorem{oss}{Observation}          
\usepackage[colorlinks=true,linkcolor=blue]{hyperref}

\begin{document}
Preliminar notes on maximum-flow problem solved in FCPP.\\
Notation:\\
 $\small\bullet\ G$ will denote the directed weighted graph of capacities, 
 $s$ and $t$ will denote respectively source and sink of our graph.\\
We will assume that capacities are nonnegative and symmetrical.\\
 $\small\bullet\ $  For a path $\mathfrak{p}$ we will denote with $|\mathfrak{p}|$ its lenght.\\

When it will be convenient, we will treat a directed weighted graph $G$ as a function $G: V\times V\ \rightarrow \mathbb{R_{+}}$, where $V$ is its set of vertices. With a little abuse of notation we will say that the edge from a node $\delta$ to a node $\delta'$ is in $G$, or $(\delta, \delta') \in G$,  in place of $G(\delta, \delta') \not= 0$.\\
Likewise we will say that a path $\mathfrak{p}$ is contained in $G$, or $\mathfrak{p}\subset G$, if every edg                                                                                                                                                                                                                                                                                                                                                                                                                                                                                                                                                                                                                                                                                                                                                                                                                                                                                                                                                                                                                                                                                                                                                                                                                                                                                                                                                                                                                                                                                                                                                                                                                                                                                                                                                                                                                                                                                                                                                                                                                                                                                                                                                                                                                                                                                                                                                                                                                                                                                                                                                                                                                                                                                                                                                                                                                                                                                                                                                                                                                                                                                                                                                                                                                                                                                                                                                                                                                                                                                                                                                                                                                                                                                                                                                                                                                                                                                                                                                                                                                                                                                                                                                                                                                                                                                                                                                                                                              

 $\small\bullet\ $ For a field $\Phi\ , \mathcal{R}_{\Phi}$ will denote the directed weighted graph of residual capacities respect to $\Phi$:
\begin{equation}
\label{eq:residual-capacity}
\mathcal{R} =
    G-\Phi
\end{equation}

$\small\bullet\ $  We define admissible path respect to $\Phi$  a path $\gamma$ in $G$ from source to sink such that 
\begin{equation*}
\Phi|_{\gamma} < G|_{\gamma}
\end{equation*}
$\small\bullet\ $ We say that a flow $\Phi$ is maximal if there are no admissible paths in $G$ respect to $\Phi$.

\begin{prop}
$\Phi$ is a maximum flow iff $\mathcal{R}_{\Phi}$ does not have paths from source to sink.\\ \\ \\ \\
\end{prop}


For a field-value $f$ in $\delta$ we define $\displaystyle{|f|:=\sum_{\delta\sim\delta'}f(\delta')}$.
We now formalize functions involved in the algorithm. Let $\Phi$ be a field and $\delta\sim\delta'$ be  devices, we define $\Phi^*(\delta, \delta'):=\Phi(\delta', \delta)$.
By co-induction:
\begin{equation*}
\begin{split}
\Phi_0 &:= \Phi\\
\mathcal{R}_n&:= \mathcal{R}_{\Phi_n}\\
e_n(\delta)&:=
\begin{cases}
\infty &\text{if\ }\delta=s\\
-\infty &\text{if\ }\delta=t\\
|\Phi_n(\delta)| &\text{otherwise}
\end{cases}\\
d_0(\delta)&:=
\begin{cases} 
0 &\text{if\ }\delta=t\\
\infty&\text{otherwise}
\end{cases}\\
d_n(\delta)&:=
\begin{cases}
0 &\text{if\ }\delta=t\\
 \mathrm{min}\big\{d_{n-1}(\delta') + 1\ |\  \mathcal{R}_{n-1}(\delta, \delta')>0\big\}&\text{otherwise}
\end{cases}\\
I_{n+1}&:= \mathrm{trunc}((G + \Phi^*_n)\cdot(d_{n-1}^*<d_n),  e(\Phi_n^*))\\
\Phi_{n+1}&:= -\Phi_n^* + I_{n+1} + \mathrm{trunc}(\Phi_n^*\ ,\  e(\Phi_n^*) - |I_{n+1}|)\\
X_n&:=\{\delta \ |\ d_n(\delta)=\infty\}
\end{split}
\end{equation*}

For a field $\Phi$ we define 
\begin{equation*}
\Vert \Phi \Vert:=\sum_{\substack{\delta \text{ connected to } \\ t    \text{ in } \mathcal{R}_{\Phi}}}|\mathcal{R}_{\Phi}(\delta)|
\end{equation*}

\begin{lem}
Let $\Phi$ be a pre-flow. Then  $\Vert \textbf{update}(\Phi) \Vert\ \leq \Vert \Phi \Vert$ with equality holding iff $\textbf{update}(\Phi)=\Phi$.
\end{lem}


\end{document}