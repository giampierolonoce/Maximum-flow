\documentclass{article}
\title{Tesi}
\author{}
\date{}
\usepackage{amsthm}
\usepackage[utf8x]{inputenc}
\usepackage{amsmath}
\usepackage{amssymb}
\usepackage{amsfonts}
\usepackage{mathrsfs}
\usepackage{graphicx}
\usepackage{floatflt}
\usepackage{amscd}
\usepackage{epigraph}
\usepackage{verbatim}
\theoremstyle{plain}                    
\newtheorem{teo}{Theorem}[section]      
\newtheorem{prop}[teo]{Proposizione}    
\newtheorem{cor}[teo]{Corollary}       
\newtheorem{lem}[teo]{Lemma}            
\theoremstyle{definition}               
\theoremstyle{remark}                  
\newtheorem{oss}{Observation}          
\usepackage[colorlinks=true,linkcolor=blue]{hyperref}

\begin{document}
Preliminar notes on maximum-flow problem solved in FCPP.\\
Notation:\\
 $\small\bullet\ G$ will denote the directed weighted graph of capacities, 
 $s$ and $t$ will denote respectively source and sink of our graph.\\
 %$\small\bullet\ $ For a directed graph $G$, $d_G(\delta, \delta')$ will be the distance from node $\delta$ to node $\delta'$. It takes values in $\mathbb{N}\cup\{\infty \}$.\\
 %$\small\bullet\  $ For a acyclic directed graph $G$, $D_G(\delta, \delta')$ will be the maximum of the lenghts of all paths from node $\delta$ to node $\delta'$.\\
 $\small\bullet\ $  For a path $\mathfrak{p}$ we will denote with $|\mathfrak{p}|$ its lenght.\\

When it will be convenient, we will treat a directed weighted graph $G$ as a function $G: V\times V\ \rightarrow \mathbb{R_{+}}$, where $V$ is its set of vertices. With a little abuse of notation we will say that the edge from a node $\delta$ to a node $\delta'$ is in $G$, or $(\delta, \delta') \in G$,  in place of $G(\delta, \delta') \not= 0$.\\
Likewise we will say that a path $\mathfrak{p}$ is contained in $G$, or $\mathfrak{p}\subset G$, if every edge in $\mathfrak{p}$ is in $G$.\\

 $\small\bullet\ $ For a field $\Phi\ , \mathcal{R}_{\Phi}$ will denote the directed weighted graph of residual capacities respect to $\Phi$:
\begin{equation}
\label{eq:residual-capacity}
\mathcal{R} =
\begin{cases}
    G+\Phi& \text{where } \Phi< 0\\
    0              & \text{otherwise}
\end{cases}
\end{equation}

$\small\bullet\ $  We define admissible path respect to $\Phi$  a path $\gamma$ in $G$ from source to sink such that 
\begin{equation*}
\Phi|_{\gamma} < G|_{\gamma}
\end{equation*}
$\small\bullet\ $ We say that a flow $\Phi$ is maximal if there are no admissible paths in $G$ respect to $\Phi$.

\begin{prop}
$\Phi$ is a maximum flow iff $\mathcal{R}_{\Phi}$ does not have paths from source to sink.
\end{prop}
\begin{proof}
Let's show that a path $\gamma$ in $G$ is admissible iff $\gamma$ is a path in $\mathcal{R}_{\Phi}$. On every edge $e$ in $G$ we have
\begin{equation*}
\begin{split}
0<{\mathcal{R}_{\Phi}}(e) &\Leftrightarrow
\begin{cases}
&\Phi(e) \leq 0 < {\mathcal{R}_{\Phi}}(e) = G(e)\ \  \ \text{or} \\
&0 <  {\mathcal{R}_{\Phi}}(e) = G(e) - \Phi(e)
\end{cases}\\
&\Leftrightarrow \Phi(e) < G(e) \ \  \text{and}\  \ 0<G(e)
\end{split}
\end{equation*}
Hence  
\begin{equation*}
0\ <\ {\mathcal{R}_{\Phi}}|_{\gamma} \Leftrightarrow\   \ \Phi|_{\gamma} < G|_{\gamma}\  \text{and}\  0< G|_{\gamma}
\end{equation*}
This is what we wanted to prove.
\end{proof}

For a field-value $f$ in $\delta$ we define $\displaystyle{|f|:=\sum_{\delta\mapsto\delta'}f(\delta')}$.

For a pre-flow $\Phi$ we define 
\begin{equation*}
\Vert \Phi \Vert:=\sum_{\substack{\delta \text{ connected to } \\ t    \text{ in } \mathcal{R}_{\Phi}}}|\mathcal{R}_{\Phi}(\delta)|
\end{equation*}

\begin{lem}
Let $\Phi$ be a pre-flow. Then  $\Vert \textbf{update}(\Phi) \Vert\ \leq \Vert \Phi \Vert$ with equality holding iff $\textbf{update}(\Phi)=\Phi$.
\end{lem}


\end{document}